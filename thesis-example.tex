\documentclass[fontsize=11pt, % size of the font
			   chapterprefix=true,
			   % oneside, % other option: oneside
			   % chapterprefix=false, % specifies if the word chapter is printed for chapter headings
			   % BCOR=0mm, % offset for binding
			   setPDF, % sets twosided=semi, BCOR=0, DIV=classic if not specified
			   % showframe, % show the page layout
			   ]{fau-math-thesis}
%
% ########################################################################################
% The following specifies the titlepage and can not be deleted
% ----------------------------------------------------------------------------------------
\title{Title of your thesis}
\author{Your Name}
\thesistype{Bachelor-Arbeit}
\degree{Bachelor of Science (B.Sc.)}
\Referee{first=Prof. A, second=Dr. B, third= MSc. C} % Referees
%\subdate{Today} % Use this if you want to specify the date on the titlepage
% ########################################################################################
\keywords{topicA, topicB, topicC} %keywords of your thesis
\setpdfinfo % For additional information in the pdf file
% ########################################################################################
% bibliography
\addbibresource{bibliography/thesis-bibliography.bib}
%

%
\usepackage{styles/fau-colors}
\colorthemefaumath
%
\usepackage[thmboxing=styleA,
			boxingstyle=styleA,
			chapterheader=styleA, 
			footerheader=styleA]{styles/fau-appearence}
\usepackage{styles/CommandSpec}

\usepackage{kantlipsum}
\usepackage{layouts}
\usepackage{tabularx}
\usepackage{graphicx}
\usepackage{todonotes}
\tcbuselibrary{listings}
\lstset
{	language={[LaTeX]TeX},
    breaklines=true,
    basicstyle=\ttfamily,
    keywordstyle=\color{blue},
    identifierstyle=\color{\colmain},
    texcsstyle=*\color{\colmain},
    commentstyle=\color{gray}
}
\newtcblisting[auto counter, number within=chapter]{lstbox}[2][]{
			   styleA, 
			   title=LISTING \thetcbcounter: #2,#1, 
			   listing options={language={[LaTeX]TeX}}}


\begin{document}
%-------------------------------------------------
\frontmatter
%-------------------------------------------------
\maketitle
\tableofcontents
\newpage
%-------------------------------------------------
\mainmatter
%-------------------------------------------------
\dedication{
This is entirely dedicated to Benno B.\\
I would have never been able to it without him!}
%++++++++++++++++++++++++++++++++++++++++++++++++++++++++++++++++++++++++++++++++++++++++
\chapter{About}\label{ch:One}
%-------------------------------------------------------------------------------
\section{Purpose of this document}
This document is a bundle of various finds and techniques that I came across with 
during the process of \textit{writing mathematics}. It both serves as an introduction 
to the programming language \LaTeX\ as well as an user guide for the class 
\texttt{\classname}. Concerning the first aspect, one has to note that there are in fact 
a number of very good tutorials and courses available online and in print \cite{latex}.
The reader is strongly encouraged to use the given sources, however the introduction 
presented here may suffice for the target audience and is a fortiori to be understand as a 
memory aid for frequent \LaTeX\ users, which was a strong motivation for me to write 
this document. Concerning my perception of the target audience, I will note the following: 
Primarily it consists of FAU students that are in the process of writing any thesis 
involving mathematics, secondarily of anybody else with the same problem. 
The topics mentioned so far cover the task of making the computer do what you want, thus 
in order to help you write a good thesis I will also present some guidelines that improve 
your wants. Finally it should be noted that the style of writing and respectively teaching 
is rather casual and informal, which is a purposely implemented feature, especially the use 
of the personal pronoun---which is highly uncommon when writing a thesis---is a product of this 
directive.
%-------------------------------------------------------------------------------
\section{What is the class \texttt{\classname}?}
If you are a beginner and didn't look into the \LaTeX\ basics yet you may skip this section. 
To answer the question of the title plain and simple, \texttt{\classname} is a \LaTeX\ class.
While there is a humongous amount of cleverly written class files available, 
especially for the purpose of writing a thesis in mathematics, one should note an 
important difference concerning the philosophy of \texttt{\classname}.
\begin{emphBox}
Simplicity over flexibility! This template is designed for a specific purpose and is 
to be customized only within the boundaries of this task.
\end{emphBox}
Many packages and class files \cite{refLatex} tend to offer very flexible solutions that 
are applicable to a vast variety of problems, the main goal of this project is to offer a 
rather intuitive \LaTeX\ template that on the one hand allows beginners to get started 
with writing mathematics and on the other hand helps students write their thesis. 
Hence, the mentioned \textit{specific purpose} is best described by
\begin{itemize}
\item learning \LaTeX,
\item writing a Bachelors or Master thesis that involves mathematics.
\end{itemize}
One has to take into account that \texttt{\classname} was not written from scratch but rather 
is a melting pot of different techniques I employed in various documents. 
Historically it bases off of my Bachelor thesis, which was a rebase of another one and so on. 
In fact, it was this frustrating process of sending code around via email to students and 
colleagues that showed the need for an unified framework. So it is indeed an aspiration 
to substitute the template that is currently provided by the mathematics department of 
FAU, \cite{l}. Especially for beginners this is hard to use and does not 
offer enough functionality, thus it is not entirely wrong to claim that \texttt{\classname} 
is a cleaned and pumped up version that is easy to handle and specifically provides 
the lacking functionality mentioned before. Furthermore it is important to note 
the didactic structure of the code, as it is also to be---and was in fact---used as \LaTeX\ 
tutorial. 
%-------------------------------------------------------------------------------
\section{Structure of this document}
As mentioned before we will cover three main topics that are briefly summarized below, 
\begin{itemize}
\item \cref{ch:latex} gives a very quick introduction to \LaTeX\ which showcases the basic 
concepts of this programming language that will allow the reader to write his own thesis,
\item \cref{ch:math} is somewhat of a documentation for the class \texttt{\classname},
\item \cref{ch:writingmath} gives an introduction to the world of \textit{writing mathematics}.
\end{itemize}
%++++++++++++++++++++++++++++++++++++++++++++++++++++++++++++++++++++++++++++++++++++++++
\chapter{How to do math}\label{ch:math}
This chapter is dedicated to the functionality of the template concerning its actual and 
inherent purpose: mathematics. 
\section{Theorem and Definitions}
In the following we will use some material from \cite{FineProp2015} to showcase the possibilities. 
The package used for theorem numbering and styling is 
\hyperlink{https://www.ctan.org/pkg/tcolorbox}{\texttt{tcolorbox}}, which is a modern and 
versatile way to create nice boxes embedded in and consistent numbering scheme. The 
commands which provide the environments for theorems, definitions etc. are defined in the 
file \textit{styles/fau-appearence.sty}. We will not fully explain how to define these commands 
within the \texttt{tcolorbox} context, but we will showcase how to use them. The following
\begin{lstlisting}[language={[LaTeX]TeX}]
\begin{theorem}{Euler 1763}{fermat}
Here could be your result.
\end{theorem}
\end{lstlisting}
will result in the output
\begin{theorem}{Euler 1763}{fermat}
Here could be your result.
\end{theorem}
The number of the theorem is assigned automatically. The second argument defines the theorem 
addition as displayed above and the third argument defines the name of the label that is used 
to reference \cref{thm:fermat}. While \LaTeX provides the basic commands for cross-referencing, 
the use of \texttt{tcolorbox} suggests the usage of the 
\hyperlink{https://ctan.org/pkg/cleveref}{\texttt{cleveref}} package that enhances some 
of the basic features and is indeed very clever. The above reference was defined by the 
command
\begin{lstlisting}[language={[LaTeX]TeX}]
\cref{thm:fermat}
\end{lstlisting}
where the prefix \texttt{thm:} was defined in the \texttt{tcolorbox} settings. The following 
environment prefix combinations are provided by \textit{styles/fau-appearence.sty}: 
\begin{itemize}
\item theorem - thm,
\item definiton - def,
\item lemma - lem,
\item corollary - cor,
\item remark - rem.
\end{itemize}
The actual appearance of the theorem can be specified by the packet option \texttt{thmboxing} 
for \texttt{fau-appearence.sty}, where this file uses
\begin{lstlisting}[language={[LaTeX]TeX}]
\usepackage[thmboxing=thmstyle_plain]{styles/fau-appearence}.
\end{lstlisting} 
.\todo{showcase some of the boxes we provide}
If you want to specify the style of theorems, definitions etc. you have to use
\begin{lstlisting}[language={[LaTeX]TeX}]
renew here plz
\end{lstlisting}

%-------------------------------------------------------------------------------
\begin{definition}{}{measure}
A mapping $\mu:2^{X}\rightarrow[0,\infty]$ is called a {\bfseries measure} on the nonempty
set $X$ provided
\begin{enumerate}[roman, ref=\thetcbcounter (\roman*)]
\item $\mu(\emptyset) = 0$ and
\item \label{en:subadd} if
\begin{align*}
A\subset \bigcup_{k\in\N}A_k,
\end{align*}
then
\begin{align*}
\mu(A)\leq \sum_{k\in\N}\mu(A_k).
\end{align*}
\end{enumerate}
\end{definition}
We can reference single items of a enumeration with the help of the enumitem package. 
For example concerning \cref{def:measure} we can add the informtion that \cref{en:subadd} is 
called subadditivity.
\begin{theorem}{MUCH WOW RESULT}{}
I'm one heckin pretty result! You gotta admit that right?\\
Look an equation
\begin{align*}
a^2+b^2=c^2.
\end{align*}
\end{theorem}

Wow i have so much to say.\todo{State what that would actually be!}\newpage

\section{B}
Hello
\section{C}
This is pretty cool section.\newpage
Do you like lewis huey and the news?
%-------------------------------------------------
\backmatter
%-------------------------------------------------
\printbibliography[type = book,
                   title = Books,
                   heading = subbibliography]
\printbibliography[type = article,
                   title = Articles,
                   heading = subbibliography]
\end{document}