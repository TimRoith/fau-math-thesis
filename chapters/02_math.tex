%++++++++++++++++++++++++++++++++++++++++++++++++++++++++++++++++++++++++++++++++++++++++
\chapter{How to do math}\label{ch:math}
This chapter is dedicated to the functionality of the template concerning its actual and 
inherent purpose: mathematics. 
\section{Theorem and Definitions}
In the following we will use some material from \cite{FineProp2015} to showcase the possibilities. 
The package used for theorem numbering and styling is 
\hyperlink{https://www.ctan.org/pkg/tcolorbox}{\texttt{tcolorbox}}, which is a modern and 
versatile way to create nice boxes embedded in and consistent numbering scheme. The 
commands which provide the environments for theorems, definitions etc. are defined in the 
file \textit{styles/fau-appearence.sty}. We will not fully explain how to define these commands 
within the \texttt{tcolorbox} context, but we will showcase how to use them. The following
\begin{lstlisting}[language={[LaTeX]TeX}]
\begin{theorem}{Euler 1763}{fermat}
Here could be your result.
\end{theorem}
\end{lstlisting}
will result in the output
\begin{theorem}{Euler 1763}{fermat}
Here could be your result.
\end{theorem}
The number of the theorem is assigned automatically. The second argument defines the theorem 
addition as displayed above and the third argument defines the name of the label that is used 
to reference \cref{thm:fermat}. While \LaTeX provides the basic commands for cross-referencing, 
the use of \texttt{tcolorbox} suggests the usage of the 
\hyperlink{https://ctan.org/pkg/cleveref}{\texttt{cleveref}} package that enhances some 
of the basic features and is indeed very clever. The above reference was defined by the 
command
\begin{lstlisting}[language={[LaTeX]TeX}]
\cref{thm:fermat}
\end{lstlisting}
where the prefix \texttt{thm:} was defined in the \texttt{tcolorbox} settings. The following 
environment prefix combinations are provided by \textit{styles/fau-appearence.sty}: 
\begin{itemize}
\item theorem - thm,
\item definiton - def,
\item lemma - lem,
\item corollary - cor,
\item remark - rem.
\end{itemize}
The actual appearance of the theorem can be specified by the packet option \texttt{thmboxing} 
for \texttt{fau-appearence.sty}, where this file uses
\begin{lstlisting}[language={[LaTeX]TeX}]
\usepackage[thmboxing=thmstyle_plain]{styles/fau-appearence}.
\end{lstlisting} 
.\todo{showcase some of the boxes we provide}
If you want to specify the style of theorems, definitions etc. you have to use
\begin{lstlisting}[language={[LaTeX]TeX}]
renew here plz
\end{lstlisting}

%-------------------------------------------------------------------------------
\begin{definition}{}{measure}
A mapping $\mu:2^{X}\rightarrow[0,\infty]$ is called a {\bfseries measure} on the nonempty
set $X$ provided
\begin{enumerate}[roman, ref=\thetcbcounter (\roman*)]
\item $\mu(\emptyset) = 0$ and
\item \label{en:subadd} if
\begin{align*}
A\subset \bigcup_{k\in\N}A_k,
\end{align*}
then
\begin{align*}
\mu(A)\leq \sum_{k\in\N}\mu(A_k).
\end{align*}
\end{enumerate}
\end{definition}
We can reference single items of a enumeration with the help of the enumitem package. 
For example concerning \cref{def:measure} we can add the informtion that \cref{en:subadd} is 
called subadditivity.
\begin{theorem}{MUCH WOW RESULT}{}
I'm one heckin pretty result! You gotta admit that right?\\
Look an equation
\begin{align*}
a^2+b^2=c^2.
\end{align*}
\end{theorem}

Wow i have so much to say.\todo{State what that would actually be!}\newpage

\section{B}
Hello
\section{C}
This is pretty cool section.\newpage
Do you like lewis huey and the news?