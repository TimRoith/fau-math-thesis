%++++++++++++++++++++++++++++++++++++++++++++++++++++++++++++++++++++++++++++++++++++++++
\chapter{Writing a math thesis with \texttt{\classname}}\label{ch:math}
This chapter is dedicated to the functionality of the template concerning its actual and 
inherent purpose: mathematics. We use material from \cite{FineProp2015} to showcase 
the possibilities.
%++++++++++++++++++++++++++++++++++++++++++++++++++++++++++++++++++++++++++++++++++++++++
\section{Page layout}
In order to write anything on a sheet of paper, you will need a basic understanding 
of typography. While there exists a strong theory behind this whole topic, we will 
not dive too deep and only give minimal introduction. One should however keep in mind that 
going freestyle on your page design greatly endangers the overall quality of your document. 
So if you're not an expert on typography, it is probably best practice to just use the 
possibilities provided by \texttt{KOMA-script} 
(especially the \href{https://ctan.org/pkg/enumitem}{\texttt{typearea}} package), 
or any package where you can be sure that 
the author has a solid typographical background. Further information on this topic may be 
found in \cite{Tschichold75, Willberg99}.
\paragraph{Page vs. paper}
In order to display our thoughts and ideas we first need some kind of canvas, which we
refer to as \textit{paper}. If you're writing a, thesis the size of this 
should almost always be of a A4 or A5 format. Once you've chosen the paper size you 
need to specify the area where you want to put stuff, which is refereed to as \textit{page}.
\begin{emphBox}
The paper does not necessarily coincide with the page.
\end{emphBox}
For simplicity---as hinted above---we will assume that the page is contained in the paper. 
At first it might seem odd, as to why we differentiate be paper and page, but there is a 
simple practical example that explains why it is not a good idea to think of the paper as 
a page, using the above terminology. Suppose you print 
out your document and want to assemble the sheets of paper in some kind of fixed bundle 
this usually happens via \textit{binding}. You can think of this as gluing small stripes 
of the paper together, thus greatly decreasing the visibility of anything printed 
specifically on this area. It is therefore useful not to think of the page as the whole paper.
\par
\paragraph{Arrangement of a page} There are four main instances placed on the page, namely:
\begin{itemize}
\item the body of text: most of the things you write will be put in here, it should 
be the dominant instance of the page,
\item the header: a field above the body, which will display information about the 
current chapter, section, etc.,
\item the footer: a field below the body, which usually displays the pagination,
\item two margins: two fields placed to the right and left of the body, of which one 
actually displays marginal notes and the other is only placed for symmetrical reasons,
\end{itemize}
see \cref{fig:play}. Having identified the main instances of our page, we now need to 
define their positions and dimensions. In order to this in a way that is consistent with 
the basic rules of typography, we employ the package \texttt{typearea}.
%-----------------------------------------------------------------------------------------
\begin{figure}
\currentpage
\printheadingsfalse
\printparametersfalse
\marginparswitchtrue
\pagedesign
\caption{The basic ingredients of our page, this graphic was created by the 
\texttt{layouts} package} \label{fig:play}
\end{figure}

%-----------------------------------------------------------------------------------------





\section{Theorems and Referencing}
In this section the term \textit{theorem} does not only refer to a theorem in a mathimatical 
sense, but in fact to definitions, lemmata, examples, remarks, etc., i.e. the environments 
you usually need to write a math thesis and want to follow some consistent numbering scheme. 
The package used for theorem numbering and styling is 
\href{https://www.ctan.org/pkg/tcolorbox}{\texttt{tcolorbox}}, which offers a modern and 
versatile way to create nice boxes around your theorems, while providing the full functionality 
of the more traditional \href{https://www.ctan.org/pkg/amsthm}{\texttt{amsthm}} package. 
The necessary commands for theorems are defined in the 
file \textit{styles/fau-appearence.sty}, so you do not have to worry about that (unless you want to). 
Hence, we will not fully explain how to define these commands, 
but we will showcase how to use them. The following code snippet
\begin{lstlisting}[language={[LaTeX]TeX}]
\begin{theorem}{Euler 1763}{fermat}
Here could be your result.
\end{theorem}
\end{lstlisting}
will result in the output
\begin{theorem}{Euler 1763}{fermat}
Here could be your result.
\end{theorem}
The number of the theorem is assigned automatically. The second argument defines the theorem 
addition as displayed above and the third argument defines the name of the label that is used 
to reference \cref{thm:fermat}. While \LaTeX provides the basic commands for cross-referencing, 
the use of \texttt{tcolorbox} suggests to employ 
\href{https://ctan.org/pkg/cleveref}{\texttt{cleveref}} package that enhances some 
of the basic features and is indeed very clever. The above reference was defined by the 
command
\begin{lstlisting}[language={[LaTeX]TeX}]
\cref{thm:fermat}
\end{lstlisting}
where the prefix \texttt{thm:} was defined in the \texttt{tcolorbox} settings. The following 
environment prefix combinations are provided by \textit{styles/fau-appearence.sty}: 
\begin{itemize}
\item theorem - thm,
\item definiton - def,
\item lemma - lem,
\item corollary - cor,
\item remark - rem.
\end{itemize}
The actual appearance of the theorem can be specified by the packet option \texttt{thmboxing} 
for \texttt{fau-appearence.sty}, for example
\begin{lstlisting}[language={[LaTeX]TeX}]
\usepackage[thmboxing=thmstyle_plain]{styles/fau-appearence}.
\end{lstlisting}
is used to create the document you are reading right now.\todo{showcase some of the boxes we provide}
It is currently not supported to simply add custom box styles. If you don't want to use the 
solutions provided by \texttt{fau-appearence.sty} use the \texttt{thmcust} option 
\begin{lstlisting}[language={[LaTeX]TeX}]
\usepackage[thmcust]{styles/fau-appearence}
\end{lstlisting}
which will not define any theorem environments.\par
The \texttt{cleveref} links are preset such that link labels are capitalized and carry 
the link in them. The link colours are set via the \texttt{hyperref} package and follow the 
defined colour scheme. This can be customized globally by 
\begin{lstlisting}[language={[LaTeX]TeX}]
\hypersetup{
	urlcolor=blue,
	citecolor=red,
	linkcolor=green}
\end{lstlisting}
but also locally for each link.
\section{Lists and Enumerations}
Another key feature you may want to use inside your thesis are lists and enumerations.
In \LaTeX\ you can simple use the \lstinline[language={[LaTeX]TeX}]|itemize| environment 
like this
\begin{lstbox}[]{Itemize}
\begin{itemize}
\item The first item,
\item[$\circ$] a second item with a different bullet type.
\end{itemize}
\end{lstbox}
For enumerations we use the
\href{https://ctan.org/pkg/enumitem}{\texttt{enumitem}} package, that provides vast 
options for customization. Take a look at the following definition taken from 
\cite{FineProp2015},
%-------------------------------------------------------------------------------
\begin{definition}{}{measure}
A mapping $\mu:2^{X}\rightarrow[0,\infty]$ is called a {\bfseries measure} on the nonempty
set $X$ provided
\begin{enumerate}[roman, ref=\thetcbcounter (\roman*)]
\item $\mu(\emptyset) = 0$ and
\item\label{en:subadd} if
\begin{equation*}
A\subset \bigcup_{k\in\N}A_k,
\end{equation*}
then
\begin{equation*}
\mu(A)\leq \sum_{k\in\N}\mu(A_k).
\end{equation*}
\end{enumerate}
\end{definition}
We can reference single items of an enumeration, for example concerning \cref{def:measure} 
we can add the information that \cref{en:subadd} is called subadditivity. 
The code that produces this enumeration looks like this:
\begin{lstbox}[listing only]{Enumerate}
\begin{enumerate}[roman, ref=\thetcbcounter (\roman*)]
\item ... % first item
\item\label{en:subadd} ... % second item
\end{enumerate} 
\end{lstbox}
The option \texttt{roman} is a preset from \texttt{fau-appearence.sty} but you can use 
any valid style provided by \texttt{enumitem} itself or define one yourself. 
The argument 
\lstinline[language={[LaTeX]TeX}]|ref=\thetcbcounter (\roman*)|
specifies how the label for 
the reference of this item should be displayed, where \lstinline[language={[LaTeX]TeX}]|\thetcbcounter| 
refers to the number of the theorem an enumeration was defined in, thus 
\lstinline[language={[LaTeX]TeX}]|\cref{en:subadd}| results in \cref{en:subadd} 
instead of \textcolor{\colmain}{Item (ii)}. An enumeration outside of a theorem has to use a different
argument for referencing.
\begin{lstbox}[]{Enumerate}
\begin{enumerate}[label=(K\theenumi), ref=MyEnum (K\theenumi)]
\item ... % first item
\item\label{en:second} ... % second item
\end{enumerate} 
\end{lstbox}
Here the reference look like this, \cref{en:second}.

\section{B}
\kant
\printinunitsof{pt}{\pagevalues}
\pagediagram
\section{C}
This is pretty cool section.\newpage
Do you like lewis huey and the news?