%++++++++++++++++++++++++++++++++++++++++++++++++++++++++++++++++++++++++++++++++++++++++
\chapter{How to do math with this template}\label{ch:math}
This chapter is dedicated to the functionality of the template concearning its actual and 
inherent purpose: math. 
\section{Theorem and Definitions}
In the following we will use some material from [] to showcase the possibilities. 
The package used for theorem numbering and styling is tcolorbox.
\begin{definition}{}{measure}
A mapping $\mu:2^{X}\rightarrow[0,\infty]$ is called a {\bfseries measure} on the nonempty
set $X$ provided
\begin{enumerate}[roman, ref=\thetcbcounter (\roman*)]
\item $\mu(\emptyset) = 0$ and
\item \label{en:subadd} if
\begin{align*}
A\subset \bigcup_{k\in\N}A_k,
\end{align*}
then
\begin{align*}
\mu(A)\leq \sum_{k\in\N}\mu(A_k).
\end{align*}
\end{enumerate}
\end{definition}
We can refernence single items of a enumeration with the help of the enumitem package. 
For example concerning \cref{def:measure} we can add the informtion that \cref{en:subadd} is 
called subadditivity.
\begin{theorem}{MUCH WOW RESULT}{}
I'm one heckin pretty result! You gotta admit that right?\\
Look an equation
\begin{align*}
a^2+b^2=c^2.
\end{align*}
\end{theorem}

Wow i have so much to say.\todo{State what that would actually be!}\newpage

\section{B}
Hello
\section{C}
This is pretty cool section.\newpage
Do you like lewis huey and the news?