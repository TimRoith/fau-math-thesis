% ========================================================================================
\chapter{Example Section for \texttt{\classname}}\label{ch:ex}
This chapter demonstrates the appearance of the thesis. 
% ++++++++++++++++++++++++++++++++++++++++++++++++++++++++++++++++++++++++++++++++++++++++
\section{Demonstration}
This is a Section.
% ........................................................................................
\subsection{Subsection A}
This is a subsection. \kant[1]
\paragraph{Paragraph A}
I am a paragraph. 
\paragraph{Paragraph B}
So am I.
% ........................................................................................
\subsection{Subsection B}
This is also a subsection.
% ++++++++++++++++++++++++++++++++++++++++++++++++++++++++++++++++++++++++++++++++++++++++
\section{Citation}
This sections demonstrates how citations looks like.
This is a book \cite{Evans15}. This is an article \cite{Ben02}. This is a 
website \cite{FAUreg}.
% ++++++++++++++++++++++++++++++++++++++++++++++++++++++++++++++++++++++++++++++++++++++++
\section{Definitions, Theorems, Remarks}
This sections demonstrates how the different theorem environments look like.
%-----------------------------------------------------------------------------------------
\begin{lemma}{Compactness in metric spaces}{compMet}
Let $X$ be a metric space and $A\subset X$, then the following statements are equivalent:
\begin{enumerate}[label=(\roman*)]
\item $A$ is relatively compact;
\item $A$ is sequentially compact;
\item $A$ is totally bounded and $\overline A$ is complete.
\end{enumerate}	
\end{lemma}
%-----------------------------------------------------------------------------------------
\begin{remark}{}{}
If $X$ is a complete metric space, we know that every closed subset $A\subset X$ is 
complete and thus the last statement in the above lemma reduces to total boundedness, 
see \cite[Lem. I.6.7]{Dunf60}.
\end{remark}
%-----------------------------------------------------------------------------------------
\begin{proof}
See, for example, \cite[Lem. I.6.15]{Dunf60}.
\end{proof}
% ----------------------------------------------------------------------------------------
The main result in the context of the weak$^\ast$ topology is stated below.
\index{topology!weak}
\index{topology!weak*}
\index{topology!strong}
% ----------------------------------------------------------------------------------------
\begin{definition}{}{}
\begin{enumerate}[label=(\roman*)]
\item Given a set $X$ and a family $\mathcal{F}$ of functions $f_i:X\rightarrow \mathcal{Y}_i$ 
associated with topological spaces $\mathcal{Y}_i$ we denote by $\sigma(X,\mathcal{F})$ 
the initial topology, i.e., the coarsest topology on $X$ such that each $F_i$ is continuous.
\item Let $X$ be a Banach space, then we denote by $\sigma(X,X^{\ast})$ the 
\emph{weak topology} on $X$, while the usual one induced by the norm is referred to as 
\emph{strong topology}.
\item Considering the family $\mathcal{F}:=\{X^{\ast}\ni\xi\mapsto\xi(x)\in\R: x\in E\}$ 
we call $\sigma(X^{\ast}, \mathcal{F}) =:\sigma(X^{\ast},X)$ the weak$^\ast$ topology.
\end{enumerate}
\end{definition}
% ----------------------------------------------------------------------------------------
% ========================================================================================
\chapter{Typeface}
This chapter demonstrates the typeface.
\kant
